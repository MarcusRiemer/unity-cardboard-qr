\documentclass[a4paper]{scrartcl}
\usepackage[utf8]{inputenc}
\usepackage[ngerman]{babel}

\subject{Praktikum Virtuelle Realität und Simulation} %TODO: richtiger name?
\title{Augmented Reality mit Google Cardboard}
\author{Mervyn McCreight \and Tim Pauls}
\date{\today}

\begin{document}
\maketitle

\section{Projektziel}
Das Ziel des Projektes ist eine Simulation von erweiterter Realität mit Hilfe eines Google-Cardboards und einem Android Handy zu erschaffen.
Eine Herausforderung dabei ist, sich ein Vorgehen zu überlegen mit dem man Augmented Reality mit dem Google-Cardboard ermöglichen kann,
da das Google Cardboard konzeptionell für das Umsetzen von virtueller Realität gedacht ist. Das Ergebnis des Projekts wird also eine ausführbare
Cardboard-kompatible Android Applikation sein.

\section{Google Cardboard}
Bei dem Google Cardboard handelt es sich um eine Initiative von Google, die das Ziel hat, Menschen virtuelle Realität zu möglichst geringen Kosten
zugängig zu machen. Grundsätzlich stellt das Cardboard einen offenen Standard zur Konstruktion eines VR-Headsets dar, in dem ein Smartphone als
Bildschirm und Bewegungsmesser eingesetzt wird.

\section{Projektinhalt}
Die Umsetzung des Projektziels erfordert vor der eigentlichen Implementierung eine Reihe von Vorüberlegungen und Nachforschungen.

\subsection{Platzierung der Erweiterungsobjekte}
Eine Herausforderung ist, dass die virtuellen Erweiterungen der Realität sich für den Betrachter scheinbar am selben Punkt befinden müssen.
Hierfür müssen die Objekte sich dynamisch mit dem Bild der Realität mitbewegen. Es gilt also einen Zusammenhang zwischen Realität und den eingefügten
Objekten herzustellen. Es gilt zu untersuchen auf welche Art und Weise sich dieses für die Zwecke dieses Projekts am Besten umsetzen lässt.

\subsection{Entwicklungsumgebung}
Weiter gilt es herauszufinden, welche Art von Entwicklungsumgebung sich am Besten dafür eignet ein solches Projekt umzusetzen.
Hierfür gibt es generell zwei verschiedene Ansätze, die es zu evaluieren gilt:
\begin{description}
  \item[natives Android]
  Entwickeln in Java mit dem Android-SDK und dem offiziellen Cardboard-SDK und OpenGL-ES.
  \item[Engines]
  Das Verwenden von Engines, die für mobile Geräte geeignet sind (z.B. Unreal Engine oder Unity3D). Hierbei gilt es auch herauszufinden
  inwiefern eine Engine mit dem Google-Cardboard kompatibel ist (Stereoskopie, etc.).
\end{description}
Grundlegende Gesichtspunkte sind hierbei der resultierende Arbeitsaufwand und die Performance zur Laufzeit.

\end{document}
