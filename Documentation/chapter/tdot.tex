\chapter{Tag der offenen Tür}

Die Entscheidung, mit Hilfe des Projektergebnisses eine Schnitzeljagd auf dem “Tag der offenen Tür” der FH-Wedel zu veranstalten, resultierte in verschiedene zusätzliche, auch innerhalb der Applikation logische, Anforderungen.
Der Ablauf der Schnitzeljagd wurde folgendermaßen entschieden:

\begin{itemize}
\item Es existieren QR-Codes, die Fragen beinhalten
\item Zu jeder Frage existiert ein QR-Code, der eine Münze beinhaltet
\item Die Münz-QR-Codes werden erst “freigeschaltet”, wenn ein Benutzer die zugehörige Frage korrekt beantwortet hat
\item Die Münzen müssen einmalig einsammelbar sein
\item Ein Benutzer hat die Schnitzeljagd gewonnen, wenn er alle Münzen eingesammelt hat
\end{itemize}
Folgende Anforderungen wurden für das Fragesystem formuliert:

\begin{itemize}
\item Die Fragen müssen dynamisch über ein Frontend ersichtlich und anlegbar sein
\item Die Applikation muss mit einem Fragen-Backend kommunizieren, um die Fragen zu erhalten und auswerten zu können
\end{itemize}

Hierfür müssen ein zusätzliches Front- und Backend entwickelt werden, dass die verschiedenen Fragen und Münzen verarbeitet, und dazu in der Lage ist für beide entsprechende QR-Codes zu generieren.

\section{Backend}
Das Backend wurde in PHP entwickelt. Die Entscheidung wurde getroffen, da der Server, auf dem das Backend laufen sollte, PHP bereits vorkonfiguriert hatte.

Das Persistieren der verschiedenen Fragen und dazugehörigen Münzen geschieht direkt über das Dateisystem. Hierfür werden für die Fragen- und Münzinformationen Dateien angelegt, die den entsprechenden Inhalt tragen. Die Informationen in den Dateien werden im JSON-Format hinterlegt.
Um die Speicherung zu strukturieren, wird automatisch folgendes Dateisystem aufgebaut, sobald eine Frage angelegt wird:

\begin{verbatim}
<backend-root>
+-- question
|   +-- 1.json
+-- coin
|   +-- 1.json
\end{verbatim}

Das Beispiel zeigt das Dateisystem in einem Zustand mit einer angelegten Frage. Man sieht, dass für Fragen und Münzen seperate Ordner anlegt werden. Zwischen Fragen und Münzen herrscht eine implizite 1:1 Verbindung. Daher ist eine Frage eindeutig über ihre ID zu einer Münze zuweisbar. Eine Münze besitzt daher immer die selbe ID wie ihre dazugehörige Frage.
Das Anlegen einer Frage und ihrer zugehörigen Münze wird in der Datei \emph{process.php} verarbeitet, die durch einen POST-Request beim Absenden des Fragenformulars im Frontend aufgerufen wird. Die zugrundelegenden Datenstrukturen und Funktionen werden in der Datei \emph{data.php} definiert. An dieser Stelle befindet sich die Logik der Serialisierung und der Speicherung im Dateisystem.

Der lesende Zugriff auf das Backend wurde über eine in PHP entwickelte RESTful-API realisiert. Die Implementierung befindet sich in der Datei \emph{api.php}.
Die RESTful-API bietet folgende Zugriffs-URLs:

 \begin{lstlisting}
GET:
/questions/	# Liefert ein JSON mit allen aktuell
		# existenten Fragen.
/questions/<id>	# Liefert ein JSON mit 
		# den Informationen zur Frage
		# der entsprechenden ID.

/qrcodes/<id>	# Liefert ein JSON mit 
		# den Aufruf-URLs der
		# Google-QRCode-API fuer Frage
		# und Muenze mit der ID (200x200).
/qrcodesprint/<id>	# Liefert ein JSON mit 
		# den Aufruf-URLs der
		# Google-QRCode-API fuer Frage
		# und Muenze mit der ID (400x400).

/questioncount/	# Gibt an, wie viele Fragen aktuell 
		# existieren.

DELETE:
/questions/<id>	# Loescht die Frage mit der 
		# angegebenen ID.
\end{lstlisting}

%\begin{lstlisting}
%GET:
%/questions                                    # Liefert ein JSON mit allen aktuell existenten Fragen.
%/questions/<id>                          # Liefert ein JSON mit den Informationen zur Frage der entsprechenden ID.
%
%/qrcodes/<id>                             # Liefert ein JSON mit den Aufruf-URLs der Google-QRCode-API für Frage und Münze mit der ID 200x200.
%/qrcodesprint/<id>                      # Liefert ein JSON mit den Aufruf-URLs der Google-QRCode-API für Frage und Münze mit der ID (400x400).
%
%/questioncount                             # Gibt an, wieviele Fragen aktuell existieren.
%
%DELETE:
%/questions/<id>                           # Löscht die Frage mit der angegebenen ID.
%\end{lstlisting}

\section{Frontend}
Das Frontend besteht aus einer mit Twitter-Bootstrap entwickelten Seite. Das Frontend kommunziert hierbei über die RESTful-API mit dem Backend. Die Seite bietet einen Überblick über alle aktuell existierenden Fragen, zusätzlich aller notwendigen Informationen (Fragetext, Antwortmöglichkeiten und korrekte Antwort).
Die Fragen werden in einer Tabelle angezeigt. Die Fragen sind hierbei nach ihrer ID sortiert. Ein Klick auf eine Fragezeile öffnet ein Modalfenster, über das es möglich ist eine Frage zu editieren, in einem speziellen Format auszudrucken oder diese unwiderruflich zu löschen. Über einen Klick auf den \emph{Neue Frage anlegen}-Button ist es möglich eine neue Frage zu erstellen.
Die für die Funktionalität relevanten Funktionen, sofern sie nicht Bootstrap-intern sind, befinden sich in der Datei \emph{overview.js}.