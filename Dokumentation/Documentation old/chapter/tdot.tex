\chapter{Tag der offenen Tür}

Die Entscheidung, mit Hilfe des Projektergebnisses eine Schnitzeljagd auf dem \glqq Tag der offenen Tür\grqq\ der FH-Wedel zu veranstalten, resultierte in verschiedene zusätzliche, auch innerhalb der Applikation logische, Anforderungen.
Der Ablauf der Schnitzeljagd wurde folgendermaßen entschieden:

\begin{itemize}
\item Es existieren QR-Codes, die Fragen beinhalten
\item Zu jeder Frage existiert ein QR-Code, der eine Münze beinhaltet
\item Die Münz-QR-Codes werden erst \glqq freigeschaltet\grqq\ , wenn ein Benutzer die zugehörige Frage korrekt beantwortet hat
\item Die Münzen müssen einmalig einsammelbar sein
\item Ein Benutzer hat die Schnitzeljagd gewonnen, wenn er alle Münzen eingesammelt hat
\end{itemize}
Folgende Anforderungen wurden für das Fragesystem formuliert:

\begin{itemize}
\item Die Fragen müssen dynamisch über ein Frontend ersichtlich und anlegbar sein
\item Die Applikation muss mit einem Fragen-Backend kommunizieren, um die Fragen zu erhalten und auswerten zu können
\end{itemize}

Hierfür müssen ein zusätzliches Front- und Backend entwickelt werden, dass die verschiedenen Fragen und Münzen verarbeitet, und dazu in der Lage ist für beide entsprechende QR-Codes zu generieren.
Zusätzlich dazu bietet das Frontend noch die Möglichkeit an, QR-Codes für Partikelsysteme zu generieren, die innerhalb der App an der Stelle des QR-Codes in die Umgebung simuliert werden können. Die veränderbaren Optionen beschränken sich zur Zeit auf die Start- und Zielfarbe der Partikel, dies lässt sich jedoch beliebig erweitern.

\section{Backend}
Das Backend wurde in PHP entwickelt. Die Entscheidung wurde getroffen, da der Server, auf dem das Backend laufen sollte, PHP bereits vorkonfiguriert hatte.

\subsection{Persistenz der Daten}
\label{sub:Persistenz}
Das Persistieren der verschiedenen Fragen und dazugehörigen Münzen, sowie der Partikelsysteme, geschieht direkt über das Dateisystem. Hierfür werden für die Fragen-/Münz- und Partikelsysteminformationen Dateien angelegt, die den entsprechenden Inhalt tragen. Die Informationen in den Dateien werden im JSON-Format hinterlegt.
Um die Speicherung zu strukturieren, wird automatisch folgendes Dateisystem aufgebaut, sobald eine Frage und/oder ein Partikelsystem angelegt wird:

\begin{verbatim}
<backend-root>
+-- question
|   +-- 1.json
+-- coin
|   +-- 1.json
+-- particle
|   +-- 1.json
\end{verbatim}

Das Beispiel zeigt das Dateisystem in einem Zustand mit einer angelegten Frage und einem angelegten Partikelsystem. Man sieht, dass für Fragen, Münzen und Partikelsysteme seperate Ordner anlegt werden. Zwischen Fragen und Münzen herrscht eine implizite 1:1 Verbindung. Daher ist eine Frage eindeutig über ihre ID zu einer Münze zuweisbar. Eine Münze besitzt daher immer die selbe ID wie ihre dazugehörige Frage. Die Partikelsysteme besitzen keine Verknüpfung zu den Fragen oder Münzen.

\subsection{API}
\label{subs:API}
Der lesende und schreibende Zugriff auf das Backend wurde über eine in PHP entwickelte RESTful-API realisiert. Die Implementierung befindet sich in der Datei \texttt{api.php}.
Die zugrundelegenden Datenstrukturen und Funktionen werden in der Datei \texttt{data.php} definiert. An dieser Stelle befindet sich die Logik der Serialisierung und der Speicherung im Dateisystem.
Die RESTful-API bietet folgende Zugriffs-URLs:

\begin{lstlisting}
POST:
/questions/         # Legt eine Frage mit den
                    # im POST-Request hinterlegten
                    # Daten an.
/particlesystems/   # Legt ein Partikelsystem mit den
                    # im POST-Request hinterlegten
                    # Daten an.

DELETE:
/questions/<id>     # Loescht die Frage mit der
                    # angegebenen ID.
/particlesystems/<id> # Loescht das Partikelsystem mit
                    # der angegebenen ID.

GET:
/questions/         # Liefert ein JSON mit allen aktuell
                    # existenten Fragen.
/questions/<id>     # Liefert ein JSON mit
                    # den Informationen zur Frage
                    # der entsprechenden ID.
/particlesystems/   # Liefert ein JSON mit allen aktuell
                    # existenten Partikelsystemen.
/particlesystems/<id> # Liefert ein JSON mit
                    # den Informationen zum Partikelsystem
                    # der entsprechenden ID.

/qrcodes/<id>       # Liefert ein JSON mit
                    # den Aufruf-URLs der
                    # Google-QRCode-API fuer Frage
                    # und Muenze mit der ID (200x200).
/qrcodesprint/<id>  # Liefert ein JSON mit
                    # den Aufruf-URLs der
                    # Google-QRCode-API fuer Frage
                    # und Muenze mit der ID (400x400).
/particleqrcode/<id>  # Liefert ein JSON mit
                    # den Aufruf-URLs der
                    # Google-QR-Code-API fuer
                    # das Partikelsystem mit der ID (200x200)
/particleqrcodeprint/<id>  # Liefert ein JSON mit
                    # den Aufruf-URLs der
                    # Google-QR-Code-API fuer
                    # das Partikelsystem mit der ID (400x400)

/questioncount/     # Gibt an, wie viele Fragen aktuell
                    # existieren.
\end{lstlisting}

\subsection{JSON-Format}
\label{sub:JSON-Format}
Es existieren verschiedene JSON-Serialisierungsformate für Fragen und QR-Code-Inhalte.

Das Format einer Frage ist folgendermaßen spezifiziert:
\begin{lstlisting}[language=JSON]
  {
    "question": "Fragetext",
    "answers": [
      "Antwort 1",
      "Antwort 2",
      "Antwort 3",
      "Antwort 4"
    ],
    "correctAnswer": 1,
    "id": 0,
    "type": 1
  }
\end{lstlisting}

Ein QR-Code-Inhalt wird verkürzt serialisiert, um den QR-Code so klein wie möglich zu halten. Dadurch beschleunigt sich die Echtzeitdekodierung. Anhand der ID wird die Verknüpfung zwischen Frage und Münze hergestellt.
\begin{lstlisting}[language=JSON]
  {
    "id": "1",
    "type": 1 // 0 = Muenze, 1 = Frage
  }
\end{lstlisting}

Für Partikelsysteme existiert eine weiteres JSON-Format, da für ein Partikelsystem zusätzlich zu seiner ID noch beliebig viele andere Zusatzdaten, die das Partikelsystem definieren, gespeichert werden müssen.
Bisher beschränken sich die konfigurierbaren Optionen auf die Start- und Zielfarbe der emitierten Partikel.

\begin{lstlisting}[language=JSON]
  {
    "id": "1",
    "type": 2 // Partikelsystem
    "startColor": "#000000" // Farben im HEX-Format.
    "endColor": "#000000" // Farben im HEX-Format.
  }
\end{lstlisting}

\section{Frontend}
Das Frontend besteht aus einer mit Twitter-Bootstrap entwickelten Seite. Das Frontend kommunziert hierbei über die RESTful-API mit dem Backend. Die Seite bietet einen Überblick über alle aktuell existierenden Fragen und Partikelsysteme, zusätzlich aller notwendigen Informationen (Fragetext, Antwortmöglichkeiten, korrekte Antwort, sowie Start- und Zielfarbe für Partikelsysteme).

Die Fragen und Partikelsysteme werden in seperaten Tabellen angezeigt. Diese sind hierbei nach ihrer ID sortiert. Ein Klick auf eine Zeile öffnet ein Modalfenster, über das es möglich ist eine Frage oder ein Partikelsystem zu editieren, in einem speziellen Format auszudrucken oder diese unwiderruflich zu löschen. Über einen Klick auf den \emph{Neue Frage anlegen}-Button oder den \emph{Neues Partikelsystem}-Button ist es möglich eine neue Frage oder ein neues Partikelsystem zu erstellen.

Die für die Funktionalität relevanten Funktionen, sofern sie nicht Bootstrap-intern sind, befinden sich in der Datei \texttt{overview.js}.

Zusätzlich zur nativen Twitter-Bootstrap Funktionalität, wurden zwei Open-Source Bootstrap-Addons verwendet. Hierbei handelt es sich zum einen um Bootstrap-Table\footnote{https://github.com/wenzhixin/bootstrap-table}, welches die Funktionalität von Tabellen in Twitter-Bootstrap erweitert, zum anderen um Bootstrap-Colorpicker\footnote{https://mjolnic.com/bootstrap-colorpicker/}, um in Bootstrap Komponenten einen HTML-Colorpicker zu integrieren.
